\documentclass[a4paper,11pt,twocolumn]{scrartcl}
%\documentclass[a4paper,12pt]{scrartcl}

% General style
\usepackage{fontspec} %LDC%
\setmainfont{Gentium Book Basic} %LDC%
\setsansfont{DejaVu Sans} %LDC%
\usepackage{newunicodechar} %LDC%
\newunicodechar{→}{\fontspec{Gentium Plus}→} %LDC%
%LDC% \DeclareUnicodeCharacter{00B1}{\pm}
\usepackage{authblk}

% Draft style
% \usepackage{setspace}
% \setstretch{1.5}
% \usepackage[top=2.5cm, bottom=2.5cm, left=2.2cm, right=3.5cm]{geometry}
% \usepackage{lineno} 

% Small style stuff
\usepackage{amsmath}

% Internal references
\usepackage[titletoc,title]{appendix}
\usepackage{hyperref}
\usepackage[capitalize]{cleveref}

\Crefname{appsec}{Appendix}{Appendices}
\crefname{appsec}{appendix}{appendices}

% Figures
\usepackage{subcaption}
\usepackage{wrapfig} %LDC%
\usepackage{graphicx} %LDC%
\usepackage{tikz}
\usetikzlibrary{arrows}
\usetikzlibrary{arrows.meta}
\usepackage[linguistics]{forest}
\usepackage{rotating}

% Tables
\usepackage{csvsimple} %LDC%
\usepackage{longtable}

% Bibliography
\usepackage[backend=biber,
            bibstyle=biblatex-sp-unified,
            citestyle=sp-authoryear-comp,
            maxcitenames=2,url=false,
            maxbibnames=99]{biblatex}
\addbibresource{acd.bib}
\renewcommand*{\bibfont}{\small}

\renewbibmacro*{doi+eprint+url}{%
    \printfield{doi}%
    \newunit\newblock%
    \iftoggle{bbx:eprint}{%
        \usebibmacro{eprint}%
    }{}%
    \newunit\newblock%
    \iffieldundef{doi}{%
        \usebibmacro{url+urldate}}%
        {}%
      }

\begin{document}
% \begin{linenumbers}
\title{Evaluating the quality of automatic cognate detection methods}
\author[1]{Gereon A. Kaiping}
\affil[1]{Leiden University Centre for Linguistics, Universiteit Leiden, NL}

\maketitle
\begin{abstract}
In this article, I compare the performance of two advanced automatic cognate detection methods, LexStat \parencite{list2012lexstat} and Online PMI \parencite{rama2017fast}. I systematically vary the most important parameters influencing the behaviour of each method. I compare the results of the automatic cognate clustering with gold-standard data annotated by historical linguists. For this comparison, I calibrate several cluster comparison measures according to the qualitative assessment of historical linguists, in order to provide a quantitative scale of coding quality that is compatible with the expectations of historical linguists.
\end{abstract}
\section{Introduction}
The challenge of identifying cognates, i.\,e. forms reflect the same common root form from an ancestral languages under different sets of sound changes towards an attested language, is an important step in the reconstruction of ancestral languages. Traditionally, identifying sound correspondences and cognates is a labour-intensive task for historical linguists.
In the last decades, automatic methods for cognate detection have been proposed and refined, becoming useful tools to support manual methods of cognate identification and other tasks in historical linguistics, for example the construction of language phylogenies \parencite{rama2018are}.

Two well-supported automatic methods developed recently are LexStat \parencite{list2012lexstat} and Online PMI \parencite{rama2017fast}. Both methods are based on reducing forms given in IPA to a limited set of sound classes and then using alignment methods to get an weighted, effective edit distance between the simplified forms. This yields a ‘cognate-ness distance’ between every pair of forms. In order to obtain cognate classes from these numbers, the complete network of all forms, with its stronger (small ‘cognate-ness distance’) and weaker (high distance) is reduced to a network in which only exactly the edges with a distance less then a threshold $θ$ are included. A standard graph clustering method then extracts those groups of forms from this graph that are highly connected to each other, but have only very few connections to forms outside the group.

A main challenge for cognate detection is to distinguish between systematic and random similarities of the forms. Systematic sound correspondences between different forms in two languages are a strong indication for shared ancestry of these forms, while short words in different languages can look similar even though they do not bear any historical relation. Cognate detection methods need to be able to account for this.

While their general structure, as described above, is similar, LexStat and Online PMI use different methods to distinguish between random and systematic similarities. LexStat normalises raw similarity scores between different forms using the similarity scores between random pairs of forms in the same languages. Online PMI processes the data in small chunks and updates its internal memory of how systematic or unsystematic sound correspondences are after each such chunk (‘online’, in the machine learning jargon), using the information-theoretic Pointwise Mutual Information measure.

Both methods are governed by a large number of parameters to account for the complexity of the process. The effect of some of these parameters has been studied in the past eg. by \textcite{list2017potential}. A comprehensive study, in particular taking the non-linear relationships between some of the parameter values into account, is still missing. A different combination of parameter values has the potential to push the quality of automatic cognate detection past the threshold of 85\% that is generally achievable with the current default values. A better understanding of the sensitivity of the methods to their parameters can further help with fine-tuning the parameters for optimal results in a specific application case.

In this article, I investigate the behaviour of LexStat and Online PMI under a
broad range of parameter values on an existing set of gold-standard datasets
that were cognate coded by historical linguists that are for the particular
language family. In \cref{acd} I will explain the two automatic cognate
detection methods and the meaning of the parameters I vary.
\Cref{lexstat,onlinepmi} contain the information specific to the respective
method. In \cref{parameters} I will summarize the parameters I investigate. In
\cref{evaluation} I will describe the methodology used to assess the quality,
where \cref{histling} contains a description of the procedure used to decide on
a cluster quality measure that would reflect historical linguists' intuitions.
In \cref{results}, I will present the results of this study, first for the
decision on cluster quality measures (\cref{measures}) and then, based on that,
of the automatic cognate detection study in \cref{clusterquality}. Lastly,
\Cref{discussion} presents a discussion of the results and my conclusions.


\section{Automatic cognate detection}
\label{acd}
As input each of the two automatic cognate detection algorithms considered here
expects word list data, where for a given set of concepts the most
representative form in each language (or sometimes multiple forms, or none) is
given in phonetic or phonemic transcription. At its core, the algorithm
calculates a score of formal similarity or “cognate-ness” for each pair of word
forms that have the same meaning. The output groups the forms into disjoint
cognate classes, each containing the forms of different languages with the same
meaning that are inferred to go back to the same root, whether through
inheritance or through borrowing.

This process consists of several steps, each of which is governed by one or more parameters.

\paragraph{Sound class}
In order to reduce the complexity of the data for the subsequent steps, and
therefore the number of parameters each method needs to estimate, the IPA
transcriptions of the forms are first translated into broader sound classes. In
this article, I test the behaviour of five sound class models available in
LingPy \parencite{lingpy26}.
\begin{description}
\item[dolgo] The idea to use a reduced set of sound classes together with
  computational alignment algorithms to automatically detect cognates goes back
  to \textcite{dolgopolsky}. The sound class model developed then is available
  as \verb|dolgo|.
\item[sca] The default model for Sound Class Alignment in LingPy, under the name
  \verb|sca|, is a modified version of the Dolgopolsky sound classes. It was
  originally introduced by \textcite{list2012sca}.
\item[asjp] The automatic similarity judgment program \parencite{asjp} has
  collected short word lists in the majority of the world's languages in a
  simplified transcription. This transcription system is designed to facilitate
  automatic comparison of forms with a focus on historical relationships and has
  therefore also been used in cognate judgment applications. A
  variant\footnote{Søren Wichmann mentioned a few differences, I have forgotten
    what they were} of this system is available as the \verb|asjp| sound class
  system.
\item[jaeger] \Textcite{jaeger} suggested a new sound class system for automatic
  cognate detection, which LingPy implements as \verb|jaeger|.
\item[\textunderscore{}color] For highlighting segmented transcriptions, eg. in web interfaces,
  LingPy also provides the \verb|_color| sound classes. As opposed to the other
  sound class models, this model is not intended for use in ACD applications,
  and I merely include it for comparison.
\end{description}

Feature-based alignment algorithms have been suggested as alternative to
soundclass-based alignment \parencite{faal}, but currently no such algorithm is
available for easy integration with ACD methods.

\paragraph{Candidate threshold}
\label{candidate}
The sound class segments of every pair of forms belonging to the same concept
are then aligned. This gives an edit distance for each pair, which represents
the number of insertions, deletions or changes that is necessary to transform
the segments of one form into the the other. This number is divided by the
length of the form\footnote{Which one? Or is it the mean, maximum, or
  something?} to obtain a normalized edit distance. Pairs of forms with a
normalized edit distance of less than a candidate threshold $η$ are assumed to
be strong candidates for cognates, and are used to calibrate the method's
scoring function.

\paragraph{Scoring pairs}
This scoring function, which is then aware of systematic and unsystematic sound
correspondences, is used to align all pairs with the same meaning, resulting in
a ‘cognate-ness distance’ score. The process of calibrating the scoring function
and calculating alignments and distance scores based on those alignments is
different between LexStat (\cref{lexstat}) and Online PMI (\cref{onlinepmi}) and
described below. Promising new scoring functions are currently being developed
\parencite{jager2016automatic,jager2017using} but not yet practically available.

The result of scoring each pair of forms is a network, in which the connections
between the forms (‘edges’ in the graph literature) are “shorter” (representing
a stronger connection) for similar forms and “longer” (representing weaker
connection) for different forms. \Cref{f:cognate-network} shows the resulting
network for the forms meaning ‘sacrifice’. Note that the length of the
connections in the figure is only an approximation of the calculated distances,
made necessary by embedding the graph in a 2-dimensional plane. The actual value
for the cognate-ness distance is instead reflected in the colour and thickness
of the lines.

\begin{figure*}
  \includegraphics[width=\textwidth]{sacrifice.pdf} %LDC%
  \caption{Network of cognate-ness for the forms meaning ‘sacrifice’, calculated using
    LexStat with the SCA sound class model. Thicker, darker lines represent a
    smaller inferred distance between forms.
    Different colours of the forms indicate different inferred cognate sets, see
    \cref{t:cognates}}
  \label{f:cognate-network}
\end{figure*}
\begin{table*}
  \centering
  \input{placeholder.tex} %LDC%
  \caption{Results of automatic cognate detection on the forms meaning
    ‘sacrifice’, for the networks shown in
    \cref{f:cognate-network,f:theta035,f:theta075}}
  \label{t:cognates}
\end{table*}

\paragraph{Clustering threshold}
The third step, common to both algorithms, is then the identification of
clusters in this weighted network. The clustering algorithm tries to find
non-overlapping sets of forms that all have a short distance to each other, but
have a long cognate-ness distance to words outside the set.

In order to execute any of the clustering algoriths, the weighted network is
first converted into an unweighted network \citeauthor{list2017potential}
according to a clustering threshold $θ$: All edges representing a cognate-ness
distance equal or greater than $θ$ are removed, the weights of the remaining
edges are ignored.

A lower threshold leaves fewer edges in the graph, potentially splitting
clusters in two, while a higher threshold leads to a more “lumping” cognate
coding. In \cref{f:infomap}, we show the same underlying data as in
\cref{f:cognate-network}, but transformed into a non-weighted graph using
different threshold values. The resulting cognate classes are reproduced in
\cref{t:cognates}. Note that in \cref{f:cognate-network}, the four forms in
different dialects of Bunak (left, dark blue) are grouped together in a single
cognate class. With a threshold of $\theta=0.35$ (\cref{f:theta035}), this class
is split in two, whereas for $\theta=0.75$ (\cref{f:theta075}) the Kula and Klon
forms are grouped together with the Bunak forms.

\paragraph{Clustering algorithm}
The resulting unweighted network is then split into clusters using a standard
graph clustering algorithm. Here, I compare the performance of four such
algorithms.

\begin{description}
\item[complete] In complete clustering, only fully-connected subgraphs are
  accepted as clusters. This is a strong condition: For a group of 6 cognate
  forms, this requires that all 15 connections between them are assigned a
  sufficiently low score to be represented in the unweighted network.
\item[single] Single linkage clustering is the opposite extreme of clustering
  algorithms: Two forms are connected if there is any path in the network
  between them. This is useful to be able to connect distantly related forms
  through intermediate relatives (eg. eng <five> – deu <fünf> – ell <pende> –
  hin <pāch>, where each supsequent pair contains similar segments, but <five>
  and <pāch> are quite dissimilar), but is highly susceptible to chance
  similarities.
\item[upgma] The Unweighted Pair Group Method with Arithmetic means
  \parencite{upgma} takes the distances on the unweighted graph and clusters the
  forms hierarchically, by taking each form as agroup of size one and then
  merging pairs of existing groups step by step until the effective distance
  between the groups becomes large. The distance of the new group to any other
  group is the Arithmetic mean of the distances of its constituent groups to
  that other group, which makes the algorithm conceptionally simple and easy to
  calculate.\footnote{For comparabililty and following
    \citeauthor{list2017potential} I use the unweighted network here, even
    though the UPGMA algorithm could in principle function directly on the
    weighted network.}
\item[infomap] In the Infomap algorithm \parencite{rosvall2008maps}, an agent
  walks randomly along the edges of the network, keeping track of which forms it
  visited. The forms visited very often are grouped together, and the agent
  continues its random walk on the remainder until all nodes have been grouped.
\end{description}

\subsection{The LexStat scoring function}
\label{lexstat}
Many current applications use the LexStat algorithm \parencite{list2012lexstat}
for cognate detection, which has been shown to perform very well for the purpose
of cognate coding \parencite{list2017potential}, and is easily available in the
LingPy software package for historical linguistics \parencite{lingpy26}. LexStat
requires at least 100 shared concepts to work well
\parencite{list2014investigating}.

LexStat \parencite{list2012lexstat} calculates systematic sound correspondences
by using the very similar pairs forms (see \cref{candidate} above) to bootstrap
a table of scores for systematic sound correspondence between each pair of
languages. The assumption behind this step is that the mismatched sound classes
between two otherwise very similar forms in two different languages are due to
systematic sound correspondences between these two languages. To account for
random similarities, these scores are normalized against random correspondences
obtained from comparing the words for random concepts in the word lists. The
cognate-ness distance of two forms can then be thought of as the effective
number of changes (discounting systematic sound changes and over-counting
unexpected correspondences, as derived from the bootstrap step) to transform one
form into the other, normalized by the chance for random similarities and the
length of the forms.

PARAMETERS


\begin{figure}
  \begin{subfigure}[b]{0.5\textwidth}
    \includegraphics[width=\textwidth]{supplement/figures/sacrifice035.pdf} %LDC%
    \caption{$\theta = 0.35$}
    \label{f:theta035}
  \end{subfigure}
  \begin{subfigure}[b]{0.5\textwidth}
    \includegraphics[width=\textwidth]{supplement/figures/sacrifice075.pdf} %LDC%
    \caption{$\theta = 0.75$}
    \label{f:theta075}
  \end{subfigure}
  \caption{Networks of cognate-ness for the forms meaning ‘sacrifice’, calculated using
    LexStat with the SCA sound class model, clustered using infomap with different thresholds $\theta$. The forms are the same as those in \cref{f:cognate-network}.}
  \label{f:infomap}
\end{figure}


\subsection{The Online PMI scoring function}
\label{onlinepmi}

An alternative to LexStat is the Online PMI method published by
\textcite{rama2017fast}. Where LexStat uses three passes over the whole dataset
to find sound correspondences in the data and how systematic they are, the
Online PMI method updates its similarity scores regularly (‘online’ in the
machine learning jargon) while going through the dataset in small batches and
adapting the scores used for future alignments after each such batch. It uses a
statistical measure for the co-occurence of sound segments known as ‘Pointwise
Mutual Information’ to generate these scores.

OnlinePMI performed better than LexStat in identifying cognate classes
\parencite{rama2017fast}, but it gave worse results when reconstructing trees
using Bayesian phylogenetics on the bases of these classes
\parencite{rama2018are}. Compared to LexStat, the Online PMI method is
supposedly better able to handle data with lower mutual coverage.

In my analysis, I use the Online PMI implementation available from
\url{https://github.com/evolaemp/online_cognacy_ident} in the version from
commit \verb#3b998ae#.

PARAMETERS

\section{Parameter values}
\label{parameters}

\begin{table*}
  \begin{tabular}{lp{0.7\textwidth}}
    \hline
    \hline
    General parameters & Values explored \\
    \hline
    Sound class & \textbf{asjp}, \textbf{sca}, dolgo, jaeger, \textunderscore{}color \\
    Candidate threshold & \textbf{0.70}, 0.50, 0.40, 0.90, 0.80, 0.30, 0.5, 0.99 \\
    Clustering threshold & \textbf{0.55}, 0.25, 0.75, 0.35, 0.45, 0.65, 0.50, 0.60\\
    Clustering algorithm & \textbf{infomap}, upgma, single, complete\\
    \hline
    \hline
    LexStat parameters \\
    \hline
    Local/global ratio & \textbf{3}, 0.5, 1, 10, 0\\
    Gap opening penalty & \textbf{-2}, -1.5, -2.5, -4, -0.5\\
    Alignment mode & \textbf{overlap}, global, local, dialign \\
    \hline
    \hline
    Online PMI parameters \\
    \hline
    Batch size & \textbf{64}, 256, 1024, 128, 512, 2048 \\
    Alpha & \textbf{0.70}, 0.50, 0.80, 0.75, 0.60 \\
    \hline
    \hline
  \end{tabular}
  \caption{Parameter values systematically explored in this study. Default values, which are
    kept fix most of the time, are given initially and in bold.}
  \label{t:parameters}
\end{table*}

The parameter exploration was conducted as a star-like search on the
combinations of different parameter values. It explores combinations of
parameter values listed in \cref{t:parameters}, starting with the default values
currently suggested by the existing literature (bold in the table), followed by
those combinations where one of the parameters differs from the default, then
two differences, and so on. In addition I explored several parameter settings
estimated as highly promising according to the model tree method described in
\cref{results} based on the results of the first 12\,000 star-search parameter
settings.

\section{Cluster comparison measures}
\label{evaluation}

In order to assess the quality of the inferred cognate codes for each parameter
value, they need to be compared with an expert-generated gold standard coding.
To make the codings for different parameter values commensurable, the result of
the comparison must be a single number summarizing the amount of similarity or
difference between the expert coding and the automatic one.

Past studies have used the B-cubed F score \parencite{}, or rarely other cluster
comparison scores. A priori, it is unclear which score might be useful for this
comparison. In fact, considering some properties of such scores \parencite{}, it
might be that B³ is particularly ill-suited for cognate coding, because it
heavily penalizes [], which is however a desired property in cognate coding,
because []. In order to optimize cognate coding for historical linguistics,
instead of according to an abstract measure, I selected the four promising
cluster quality measures described below. For each pair of measures, I extracted
from the parameter exploration a pair of parameters and their cognate detection
results, such than according to one cluster quality measure, the first cognate
detection result was much better than the second, whereas according to the other
cluster quality measure, the second result was much better then the first.

\subsection{Forced choice task for historical linguists}
\label{histling}
I presented colleagues with a solid background in the historical linguistics of
one of the language families available as gold standard with the six pairs of
automatic cognate coding results and asked them to judge which of two coding
results they considered better. The detailed instructions, as well as the coding
results used and the method used to select the pairs, are given in the
supplementary material.

A consistent patterns of preference for one clustering quality measure over
another one would mean that that measure would be closer to, and therefore a
useful approximation for, expert intuitions concerning the quality of cognate
judgments.

\section{Results}
\label{results}
In order to evaluate the quality of cognate coding, we have to first consider
the quality of cluster comparison methods according to the target audience of
historical linguists. Their preferred cluster comparison method will then be
used in the actual assessment of the different methods and their parameter
values.

\section{Agreement between cluster measures and historical linguists' idea of
  quality}
\label{measures}
\begin{table*}
  \begin{tabular}{llcccccc}
    Linguist & Language family & B³ vs. AMI & Overall preference order \\
    \hline
    L1 & Austronesian & B³ & B3 > AMI \\
    L2 & Indo-European & AMI & \\
    \hline
  \end{tabular}
  \caption{Preferences in the forced choice task, for the sample of historical linguists}
  \label{t:preferences}
\end{table*}
The results of the forced-choice task presented to the historical linguists are
given in \cref{t:preferences}. The overall preference is clearly in favour of
the B-cubed F score measure, so I will report the results of the cognate coding
algorithm in terms of that measure.

\section{Quality of cognate detection algorithms}
\label{clusterquality}
Data
Decision tree

\section{Discussion and conclusions}
\label{discussion}

How did the other guys get up to 85\%?

Recommendations on method and parameter values

open question (cf. Rama): How well does linguists' intuition align with quality
needed for building trees?

\printbibliography

\end{document}

% Local Variables:
% TeX-engine: xetex
% End:
